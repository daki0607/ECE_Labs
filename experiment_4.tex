\documentclass[12pt]{article}
\usepackage{amsmath}
\usepackage{bm}

\renewcommand{\contentsname}{Table of Contents}

\begin{document}

\pagenumbering{roman}
\title{Experiment 4.2\\Maximum Power Transfer}
\author{Justin B., Aditya D.}
\date{\today}
\maketitle
\pagebreak

\tableofcontents
\pagebreak
\pagenumbering{arabic}

\section{Objectives}
When designing circuits, the maximum power transfer to a particular component may be a critical part of the design.
In this experiment, we measure and model an unknown circuit as its Thevenin equivalent at steady state.
Next, we design the load that draws the maximum power under certain constraints.

\section{Experimental Results}
We set the signal generator to output a voltage of \(\bm{V} = 7.00\angle 0^\circ V\), as suggested by the technician, to accomodate for the line and internal impedances by the time the signal gets to the Black Box.
We measure \(V_{OC} = 0.835 V\), \(V_{OC, rms} = 2.78 V\), and \(\theta_v = 35^\circ\).

We connect a resistive load \(R_L = 9.92\Omega\) and measure \(I_{L, rms} = 43.64mA\).
The resulting phase is measured as \(\theta_v = \theta_i = 19^\circ\).
Using the magnitude and phase, the load current phasor is \(\bm{I}_L = 0.04364\angle 19^\circ A\).

Using the open circuit voltage phasor and the load current phasor, the Thevenin impedance is calculated as follows:

\begin{align*}
    \bm{Z}_{TH} &= \frac{\bm{V}_{OC}}{\bm{I}_L} \\
                &= \frac{0.835\angle 35^\circ V}{0.04364\angle 19^\circ A} \\
                &= 9.929\angle 32^\circ \Omega.
\end{align*}

\section{Discussion}
\section{Conclusion}

\newpage
\addcontentsline{toc}{section}{References}
\begin{thebibliography}{1}
    \bibitem{labManual}
    N. Dimopoulos, F. Gebali, \textit{Laboratory Manual: ECE 250, \\ Linear Circuits I (Edition 4)}, University of Victoria, Victoria, B.C, 2018, pp.46-49.
    
\end{thebibliography}



\end{document}