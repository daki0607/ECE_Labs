\documentclass[12pt]{article}
\usepackage{amsmath}
\usepackage{bm}

\renewcommand{\contentsname}{Table of Contents}
\setlength{\parskip}{1em}

\begin{document}

\pagenumbering{roman}
\title{Experiment 4.2\\Maximum Power Transfer}
\author{Justin B, Aditya D.}
\date{\today}
\maketitle
\pagebreak

\tableofcontents
\pagebreak
\pagenumbering{arabic}

\section{Objectives}
When designing circuits, the maximum power transfer to a particular component may be a critical feature, so we must account for this factor in our design.
In this experiment, we measure and model an unknown circuit as its Thevenin equivalent at steady state.
Next, we design the load that draws the maximum power under certain constraints \cite{labManual}.

\section{Experimental Results}
We set the signal generator to output a voltage of \(\bm{V} = 7.00\angle 0^\circ V\), as suggested by the technician, to accomodate for the line and internal impedances by the time the signal gets to the Black Box.
We measure \(V_{OC} = 0.835 V\), \(V_{OC, rms} = 2.78 V\), and \(\theta_v = 35^\circ\).

We connect a resistive load \(R_L = 9.92\Omega\) and measure \(I_{L, rms} = 43.64mA\).
The resulting phase is measured as \(\theta_v = \theta_i = 19^\circ\).
Using the magnitude and phase, the load current phasor is \(\bm{I}_L = 0.04364\angle 19^\circ A\).

Using the open circuit voltage phasor and the load current phasor, the Thevenin impedance is calculated as follows:
\begin{align*}
    \bm{Z}_{TH} &= \frac{\bm{V}_{OC}}{\bm{I}_L} \\
                &= \frac{0.835\angle 35^\circ V}{0.04364\angle 19^\circ A} \\
                &= 9.929\angle 32^\circ \Omega.
\end{align*}

With the available components, we wired 7 capacitors with \(C_L = 490\mu F\) in parallel which resulted in \(R_L = 13.6\Omega\) for maximum power transfer.
\newpage

\begin{table}[h!]
    \centering
    \begin{tabular}{c|c||c|c}
    \(R_L (\Omega)\) & \(P (W)\) & & \\ \hline
    20           & 19.0 & 10 & 24.0 \\
    19           & 19.5 & 9 & 24.4 \\
    18           & 20.0 & 8 & 24.8 \\
    17           & 20.5 & 7 & 25.0 \\
    16           & 21.0 & 6 & 25.0 \\
    15           & 21.5 & 5 & 24.7 \\
    14           & 22.0 & 4 & 23.8 \\
    13           & 22.6 & 3 & 22.2 \\
    12           & 23.1 & 2 & 19.5 \\
    11           & 23.6 & 1 & 14.5 \\
    \end{tabular}
    \caption{Real power vs Load resistance}
    \label{table:1}
\end{table}

Varying the load resistance, we obtain the following powers as shown in Table \ref{table:1}.

\section{Discussion}
As calculated in the Experimental Results, we obtain a Thevenin equivalent of:
\begin{align*}
    \bm{V}_{Th} = 0.835\angle 35^\circ \text{ and } \bm{Z}_{Th} = 9.929\angle 32^\circ
\end{align*}

For maximum power transfer to occur, \(\bm{Z}_{L} = \bm{Z}_{Th}^*\), however, due to component restrictions, \(X_L = -5.4133\).
Therefore, 
\begin{align*}
    R_L &= \sqrt{R_{Th}^2 + \left(X_{Th} - X_L\right)^2} \\
        &= \sqrt{(8.42)^2 + \left((5.26) - (-5.4133)\right)^2} \\
        &= 13.6\Omega \text{ (Our estimate)}
\end{align*}

As seen in Table \ref{table:1}, the resistance that drew the maximum power was around \(7\Omega\).
When measured, a final value of \(R_L = 7.2\Omega\) was obtained to maximize power delivered to the load.

The true value of \(R_L\) was lower than expected, which may have been due to unknown impedances in wires, the Black Box, and the equipment used.
We theorize that the many sources of impedance combine to form a voltage divider that reduces the overall power drawn by the load.

\section{Conclusion}
The constructed load demonstrated maximum power transfer to within a certain degree of error.
By measuring current and voltage of an unknown Black Box circuit, its thevenin equivalent aids in the calculation and realization of the load that draws the maximum power.

\newpage
\addcontentsline{toc}{section}{References}
\begin{thebibliography}{1}
    \bibitem{labManual}
    N. Dimopoulos, F. Gebali, \textit{Laboratory Manual: ECE 250, \\ Linear Circuits I (Edition 4)}, University of Victoria, Victoria, B.C, 2018, pp.46-49.
\end{thebibliography}



\end{document}
