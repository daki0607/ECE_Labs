\documentclass[12pt]{article}

\usepackage{amsmath}

\renewcommand{\contentsname}{Table of Contents}

\begin{document}

\title{Experiment 2.2 \\ Wheatstone Bridge}
\author{Justin B, Aditya D}
\date{\today}
\maketitle

\pagenumbering{roman}
\newpage
\tableofcontents
\newpage 
\pagenumbering{arabic}

\section{Objectives}
\section{Experimental Results}
As labeled in Figure 12 \cite{labManual}, the following values were chosen:
\begin{align*}
    E   &= 12.0V \\
    R_1 &= 100\Omega \\
    R_2 &= 910\Omega \\
    R_3 &= 0\!-\!200\Omega \\
    R_x &= 960.7\Omega 
\end{align*}
The \(200\Omega\) potentiometer was chosen because the bridge is balanced when it has a resistance of \(106\Omega\), allowing for maximal variation.

\begin{table}[h]
\centering
\begin{tabular}{c | c }
    Force (\(N\)) & Voltage (\(V\)) \\
    \hline
    1.226 & 0.167 \\
    1.275 & 0.176 \\
    1.271 & 0.182 \\
    1.238 & 0.180 \\
    1.245 & 0.176 \\
\end{tabular}
\caption{Force versus Voltage across bridge}
\label{table:1}
\end{table}

\section{Discussion}
\section{Conclusion}

\newpage
\addcontentsline{toc}{section}{References}
\begin{thebibliography}{1}
    \bibitem{labManual}
    N. Dimopoulos, F. Gebali, \textit{Laboratory Manual: ECE 250, \\ Linear Circuits I (Edition 4)}, University of Victoria, Victoria, B.C, 2018.
\end{thebibliography}

\end{document}